%%%% fatec-article.tex, 2024/03/10

%% Classe de documento
\documentclass[
landscape,
  a4paper,%% Tamanho de papel: a4paper, letterpaper (^), etc.
  12pt,%% Tamanho de fonte: 10pt (^), 11pt, 12pt, etc.
  english,%% Idioma secundário (penúltimo) (>)
  brazilian,%% Idioma primário (último) (>)
]{article}

%% Pacotes utilizados
\usepackage[]{fatec-article}
\usepackage{setspace}
\usepackage{float}

%% Processamento de entradas (itens) do índice remissivo (makeindex)
%\makeindex%

%% Arquivo(s) de referências
%\addbibresource{fatec-article.bib}

%% Início do documento
\begin{document}

% Seções e subseções
%\section{Título de Seção Primária}%

%\subsection{Título de Seção Secundária}%

%\subsubsection{Título de Seção Terciária}%

%\paragraph{Título de seção quaternária}%

%\subparagraph{Título de seção quinária}%

%\section*{Diário de Bordo}%

\begin{table}[H]
\centering
\renewcommand{\arraystretch}{1.5} % 
\resizebox{\textwidth}{!}{%
\begin{tabular}{|l|l|l|l|>{\raggedright\arraybackslash}p{7cm}|}
\hline
Nome da Atividade & Data de início & Data de término & Responsável pela atividade & Descrição da atividade realizada \\ \hline
Versão Inicial Artigo & 17/02/2025 & 14/04/2025 & Vinicius Souza & Elaboração da primeira versão do artigo acadêmico, Introdução, objetivos e estado da arte.\\ \hline
Prototipapação Figma & 17/02/2025 & 12/05/2025 & Paulo Cesar & Criação de protótipos funcionais no Figma para representar o fluxo de navegação e a interface do usuário da aplicação, garantindo uma experiência intuitiva e consistente \\ \hline
KanBan & 17/02/2025 & 12/05/2025 & Vinicius Souza, Joao Kuzinor, Paulo Cesar & Implementação e atualização do quadro Kanban para acompanhamento das tarefas do projeto.\\ \hline
Front-End (Aplicação Mobile) & 15/04/2025 & 26/05/2025 & Vinicius Souza & Desenvolvimento da interface do usuário (front-end), criando layouts, interatividade e integração com a lógica do sistema.\\ \hline
\end{tabular}
}
\end{table}

\begin{table}[]
\centering
\renewcommand{\arraystretch}{1.5} % 
\resizebox{\textwidth}{!}{%
\begin{tabular}{|l|l|l|l|>{\raggedright\arraybackslash}p{7cm}|}
\hline
Diagrama do Banco de Dados & 01/04/2025 & 12/05/2025 & Vinicius Souza & Desenvolvimento do diagrama de banco de dados, representado em coleções.\\ \hline
Desenvolvimento API & 16/04/2025 & 19/05/2025 & Vinicius Souza & Desenvolvimento e implementação da API, incluindo a criação de rotas, endpoints e integração com o banco de dados.\\ \hline
Diagrama de casos de uso & 13/04/2025 & 13/04/2025 & João Lima & Criação do diagrama de casos de uso, representando as interações entre os usuários e o sistema, destacando as funcionalidades principais.\\ \hline
Diagrama de objeto & 13/04/2025 & 13/04/2025 & João Lima & Elaboração do diagrama de objetos, ilustrando as instâncias dos objetos e suas relações durante a execução do sistema.\\ \hline
Diagrama de classe & 13/04/2025 & 13/04/2025 & João Lima & Desenvolvimento do diagrama de classes, representando as principais classes do sistema, seus atributos, métodos e relacionamentos.\\ \hline
\end{tabular}
}
\end{table}


\begin{table}[]
\centering
\renewcommand{\arraystretch}{1.5} % 
\resizebox{\textwidth}{!}{%
\begin{tabular}{|l|l|l|l|>{\raggedright\arraybackslash}p{7cm}|}
\hline
Front-End (Aplicação Web)& 16/04/2025 & 19/05/2025 & Paulo Cesar e João Lima& Desenvolvimento da interface do usuário (front-end), criando layouts, interatividade e integração com a lógica do sistema.\\ \hline
Canvas & 05/05/2025 & 05/05/2025 & Vinicius Souza & Criação do modelo Canvas, detalhando os principais elementos do projeto, como proposta de valor, segmentação de clientes e fontes de receita.\\ \hline
IoT & 23/10/2025 & 19/05/2025 & João Lima, Vinicius Souza & Configuração do arduino (ESP32 DevKit V1 - ESP-WROOM-32) que será utilizado na coleira. \\ \hline
Pitch & 10/05/2025 & 12/05/2025 & Paulo Cesar & Edição de video apresentando nosso projeto em 2 minutos. \\ \hline
Diagrama e Especificação da Infraestrutura da Rede& 17/02/2025 & 17/02/2025 & Vinicius Souza & Criação do diagrama e definição das especificações técnicas da infraestrutura de rede para o projeto.\\ \hline
Versão Final Artigo & 17/02/2025 & 12/05/2025 & Vinicius Souza & Revisão e finalização do artigo acadêmico, adicionando resultados e conclusao e ajustando a estrutura, conteúdo e formatação conforme as diretrizes.\\ \hline
\end{tabular}
}
\end{table}


\end{document}