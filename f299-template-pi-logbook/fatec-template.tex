%%%% fatec-article.tex, 2024/03/10

%% Classe de documento
\documentclass[
landscape,
  a4paper,%% Tamanho de papel: a4paper, letterpaper (^), etc.
  12pt,%% Tamanho de fonte: 10pt (^), 11pt, 12pt, etc.
  english,%% Idioma secundário (penúltimo) (>)
  brazilian,%% Idioma primário (último) (>)
]{article}

%% Pacotes utilizados
\usepackage[]{fatec-article}
\usepackage{setspace}
\usepackage{float}

%% Processamento de entradas (itens) do índice remissivo (makeindex)
%\makeindex%

%% Arquivo(s) de referências
%\addbibresource{fatec-article.bib}

%% Início do documento
\begin{document}

% Seções e subseções
%\section{Título de Seção Primária}%

%\subsection{Título de Seção Secundária}%

%\subsubsection{Título de Seção Terciária}%

%\paragraph{Título de seção quaternária}%

%\subparagraph{Título de seção quinária}%

%\section*{Diário de Bordo}%

\begin{table}[H]
\centering
\renewcommand{\arraystretch}{1.5} % 
\resizebox{\textwidth}{!}{%
\begin{tabular}{|l|l|l|l|>{\raggedright\arraybackslash}p{7cm}|}
\hline
Nome da Atividade & Data de início & Data de término & Responsável pela atividade & Descrição da atividade realizada \\ \hline
Versão Inicial Artigo & 06/10/2024 & 07/10/2024 & Vinicius Souza & Elaboração da primeira versão do artigo acadêmico, Introdução, objetivos e estado da arte.\\ \hline
Prototipapação Figma & 06/10/2024 & 28/11/2024 & Paulo Cesar & Criação de protótipos funcionais no Figma para representar o fluxo de navegação e a interface do usuário da aplicação, garantindo uma experiência intuitiva e consistente \\ \hline
Diagrama e Especificação da Infraestrutura da Rede& 23/10/2024 & 23/10/2024 & Vinicius Souza & Criação do diagrama e definição das especificações técnicas da infraestrutura de rede para o projeto.\\ \hline
KanBan & 22/10/2024 & 18/11/2024 & Vinicius Souza, Joao Kuzinor, Paulo Cesar & Implementação e atualização do quadro Kanban para acompanhamento das tarefas do projeto.\\ \hline
Scrum & 06/10/2024 & 18/11/2024 & Vinicius Souza, Joao Kuzinor, Paulo Cesar & Aplicação das cerimônias do Scrum para planejamento, acompanhamento e revisão do projeto.\\ \hline
Modelo Conceitual & 30/10/2024 & 31/10/2024 & Vinicius Souza, Joao Kuzinor, Paulo Cesar & Desenvolvimento do modelo conceitual, representando as coleções principais no projeto.\\ \hline
\end{tabular}
}
\end{table}

\begin{table}[]
\centering
\renewcommand{\arraystretch}{1.5} % 
\resizebox{\textwidth}{!}{%
\begin{tabular}{|l|l|l|l|>{\raggedright\arraybackslash}p{7cm}|}
\hline
Modelo Logico & 30/10/2024 & 31/10/2024 & Vinicius Souza, Joao Kuzinor, Paulo Cesar & Elaboração do modelo lógico, detalhando a estrutura e organização dos dados para o projeto.\\ \hline
Desenvolvimento API & 30/10/2024 & 18/10/2024 & Vinicius Souza & Desenvolvimento e implementação da API, incluindo a criação de rotas, endpoints e integração com o banco de dados.\\ \hline
Diagrama de casos de uso & 04/11/2024 & 11/11/2024 & Vinicius Souza, João Lima & Criação do diagrama de casos de uso, representando as interações entre os usuários e o sistema, destacando as funcionalidades principais.\\ \hline
Diagrama de objeto & 04/11/2024 & 11/11/2024 & Vinicius Souza, João Lima & Elaboração do diagrama de objetos, ilustrando as instâncias dos objetos e suas relações durante a execução do sistema.\\ \hline
Diagrama de classe & 04/11/2024 & 11/11/2024 & Vinicius Souza, João Lima & Desenvolvimento do diagrama de classes, representando as principais classes do sistema, seus atributos, métodos e relacionamentos.\\ \hline
Versão Final Artigo & 11/11/2024 & 17/11/2024 & Vinicius Souza & Revisão e finalização do artigo acadêmico, adicionando resultados e conclusao e ajustando a estrutura, conteúdo e formatação conforme as diretrizes.\\ \hline
\end{tabular}
}
\end{table}


\begin{table}[]
\centering
\renewcommand{\arraystretch}{1.5} % 
\resizebox{\textwidth}{!}{%
\begin{tabular}{|l|l|l|l|>{\raggedright\arraybackslash}p{7cm}|}
\hline
Front-End (Aplicação Web)& 23/10/2024 & 28/11/2024 &	Paulo Cesar e João Lima& Desenvolvimento da interface do usuário (front-end), criando layouts, interatividade e integração com a lógica do sistema.\\ \hline
Back-End (Aplicação web)& 10/11/2024 & 28/11/2024 & Vinicius Souza & Desenvolvimento da lógica de servidor (back-end), incluindo a criação de funcionalidades, integração com banco de dados e gerenciamento de requisições.\\ \hline
Canvas & 23/10/2024 & 23/10/2024 & João Lima, Paulo Cesar e Vinicius Souza & Criação do modelo Canvas, detalhando os principais elementos do projeto, como proposta de valor, segmentação de clientes e fontes de receita.\\ \hline
\end{tabular}
}
\end{table}


\end{document}